

% Les packages utilise's ci-dessous le sont a` titre indicatif ;
% vous pouvez les changer a` votre convenance.

% Le type de document: article, rapport...
\documentclass[a4paper]{report}

% Mettre les diffe'rents packages et fonctions que l'on utilise
%\usepackage[english]{babel}
\usepackage[french]{babel}
\usepackage{amsmath}
\usepackage{amssymb}
\usepackage{graphics,color}

% Commenter l'une de ces deux lignes
%\RequirePackage[applemac]{inputenc}
\RequirePackage[latin1]{inputenc}

\begin{document}


%----------- A   C O M P L E T E R   P A R   L E S   A U T E U R S ------------


% Titre du rapport
\def\TitreRapport{
    Titre du rapport \\
    pouvant �tre sur plusieurs lignes \\
    si n\'ecessaire
}

% Pre'nom et nom dde l'auteur
\def\NomsAuteurs{
    Pr\'enom Nom
}

% Date du rapport (dans la me^me langue que le titre)
\def\DateRapport{
    jour mois ann\'ee
}

% Nom des encadrants
\def\Encadrants{
    \textbf{Encadrant(s)} \\
    mettre ici les noms des encadrants
}
% Nom du laboratoire
\def\Labo{
    mettre ici le nom  du laboratoire, par d�faut le LIF
}



% Re'sume' en franc,ais avec mots-cle's
\def\ResumeFrancais{
    R\'esum\'e en fran\c{c}ais obligatoire.
    \\[2mm]
    {\bf Mots-cl\'es : } mot-cl\'es  obligatoires.
}


\thispagestyle{empty}
\begin{center}
\baselineskip=1.3\normalbaselineskip
{\bf\Large \TitreRapport}\\[8mm]
{\bf\large \NomsAuteurs}\\[1mm]
{\Labo}\\[4mm]
\Encadrants\\[10mm]

{\bf R\'esum\'e}
\end{center}

\ResumeFrancais\\[4mm]

\newpage

%-------------------- T E X T E   D U   R A P P O R T -------------------------


Ins\'erer ici le texte du rapport dans le format et la structure que
vous voulez.


N'oubliez pas de mettre \`a la fin du rapport la bibliographie.

\bigskip

Une fois que votre rapport est pr\^et, produisez-en une version
pdf. Relisez-la avec attention. V\'erifiez qu'il n'y a pas de
probl\`emes d'accentuation.



\bigskip

Merci d'avoir bien voulu suivre ces instructions.

\end{document}

